\chapter*{Aufgabe 2}

\begin{tabular}{p{.45\textwidth} p{.45\textwidth}} 
	\hline
	Use Case & Beschreibung \\ 
	\hline
	Name des Anwendungsfalls & Kaufen von Tickets \\
	\hline
	Kurzbeschreibung des Anwendungsfallziels & Ein/e Kund/in kauft Tickets einer Veranstaltung \\
	\hline 
	Goal in Context & Ein/e Kund/in kauft Tickets einer Veranstaltung \\
	\hline
	Scope & Die Anwendung „TicketNow“ \\
	\hline
	Level & Benutzerziel \\
	\hline
	Stakeholders and Interests & - Kund/in: Möchte Ticket(s) einer Veranstaltung kaufen \\
	                             & - Ticketagentur: Verkauf der Tickets, Konsistenter Datenbestand \\
	\hline
	Minimal Guarantees & Abbuchung erfolgt nur, wenn Verkauf erfolgreich war. \\
	\hline
	Success Guarantees 	& -System zeigt Zusammenfassung der Veranstaltung und Bestellformular an \\
						& -System überprüft Bestellung, und Überprüfung ist erfolgreich. \\
						& -Zusenden der Tickets, Aktualisierung des Ticketbestandes und Abbuchung des geschuldeten Betrages\\

	
	
%	
%	& - Aktualisieren der Bestände\\
%	                   & - Anzeigen des erfolgreichen Kaufabschlusses \\
%	                   & - Abbuchung des fälligen Betrages \\
	\hline
	Primary Actor & Kunde \\
	\hline
	Precondition & - Kunde lässt sich Ticketübersicht anzeigen \\
	             & - Kunde beschließt Kauf \\

	             \hline
	Main Success Scenario & 1. Kunde entschließt sich zum Kauf und wählt Aktion Kaufen aus \\
						& 2. Das Bestellformular wird ausgefüllt und abgeschickt.\\
						& 3. Prüfung der Bestellung erfolgreich.\\
						& 4. Verkauf wird ausgeführt(Zusenden des Tickets,aktualisieren der Bestände, Abbuchen des Betrags)\\
						& 5. Anzeigen des erfolgreichen Kaufabschlusses.\\
	\hline
	Extensions & 2a: Bis zur Bestätigung kann der Kaufvorgang jederzeit abgebrochen werden.\\
	           & 2b: Wird der Kauf abgebrochen, wird wieder die Veranstaltungsübersicht angezeigt.\\
	           & 3a: Ist die Überprüfung nicht erfolgreich, so wird der Kaufvorgang abgebrochen.\\
	           & 3b: Es wir in einer Meldung der Grund für die fehlgeschlagene Überprüfung angezeigt, und die Seite mit dem Bestellformular wird angezeigt. \\
	\hline           
\end{tabular} 