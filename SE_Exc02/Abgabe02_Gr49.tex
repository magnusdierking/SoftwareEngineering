%!TEX TS-program = lualatex
\documentclass[
	ngerman,
	twoside,
	pdfa=false,
	ruledheaders=section,%Ebene bis zu der die Überschriften mit Linien abgetrennt werden, vgl. DEMO-TUDaPub
	class=report,% Basisdokumentenklasse. Wählt die Korrespondierende KOMA-Script Klasse
	thesis={type=Übung},% Dokumententyp Thesis, für Dissertationen siehe die Demo-Datei DEMO-TUDaPhd
	accentcolor=TUDa-9d,% Auswahl der Akzentfarbe
	custommargins=false,% Ränder werden mithilfe von typearea automatisch berechnet
	marginpar=false,% Kopfzeile und Fußzeile erstrecken sich nicht über die Randnotizspalte
	%BCOR=5mm,%Bindekorrektur, falls notwendig
	parskip=half-,%Absatzkennzeichnung durch Abstand vgl. KOMA-Sript
	fontsize=11pt,%Basisschriftgröße laut Corporate Design ist mit 9pt häufig zu klein
%	logofile=tuda_logo.pdf, %Falls die Logo Dateien nicht installiert sind
]{tudapub}

% Sprachanpassung
\usepackage[english, main=ngerman]{babel}
\usepackage[autostyle]{csquotes}
\usepackage{microtype}
\usepackage{listings}

% Literaturverzeichnis
\usepackage{biblatex}
\addbibresource{literature.bib}

% Pakete-Mathematik & mehr
\usepackage{mathtools}
\usepackage{amsmath}
\usepackage{amsfonts}
\usepackage{subcaption}


% neu cmds
\newcommand*\diff{\mathop{}\!\mathrm{d}}
\newcommand*\Diff[1]{\mathop{}\!\mathrm{d^#1}}


\begin{document}
	\title{Software Engineering Übung 02}
	\subtitle{Anwendungsfallmodellierung}
	\author[J.Lippert \and M. Dierking]
	{Jonathan Lippert \and Magnus Dierking}
	%\reviewer{}
	%\department{ce}

	
	\submissiondate{27.11.2020}
	

	\maketitle
	\pagenumbering{gobble}


	% Kurzzusammenfassung
%	\include{chapters/zusammenfassung}

	% Inhaltsverzeichnis 
%	\tableofcontents
	\newpage
	\pagenumbering{arabic}
	\setcounter{page}{1}

%-------------------------------------------------------------------------------------------------
     \chapter*{Aufgabe 1}
     \chapter*{Aufgabe 2}

\begin{tabular}{p{.45\textwidth} p{.45\textwidth}} 
	\hline
	Use Case & Beschreibung \\ 
	\hline
	Name des Anwendungsfalls & Kaufen von Tickets \\
	\hline
	Kurzbeschreibung des Anwendungsfallziels & Ein/e Kund/in kauft Tickets einer Veranstaltung \\
	\hline 
	Goal in Context & Ein/e Kund/in kauft Tickets einer Veranstaltung \\
	\hline
	Scope & Die Anwendung „TicketNow“ \\
	\hline
	Level & Benutzerziel \\
	\hline
	Stakeholders and Interests & - Kund/in: Möchte Ticket(s) einer Veranstaltung kaufen \\
	                             & - Ticketagentur: Verkauf der Tickets, Konsistenter Datenbestand \\
	\hline
	Minimal Guarantees & Abbuchung erfolgt nur , wenn Kauf bestätigt wurde \\
	\hline
	Success Guarantees & - Aktualisieren der Bestände\\
	                   & - Anzeigen des erfolgreichen Kaufabschlusses \\
	                   & - Abbuchung des fälligen Betrages \\
	\hline
	Primary Actor & Kunde \\
	\hline
	Precondition & - Kunde lässt sich Ticketübersicht anzeigen \\
	             & - Kunde beschließt Kauf \\
	             & - Kunde füllt Bestellformular aus \\ 
	             & - Kunde bestätigt Kauf \\
	             \hline
	Main Success Scenario &  Kunde wählt Aktion "Kaufen" aus \\
	\hline
	Extensions & 2a: \\
	           & 2b: \\
	           & 3a: \\
	           & 3b: \\
	\hline           
\end{tabular} 
	
%-------------------------------------------------------------------------------------------------

	% Literaturverzeichnis
	\printbibliography % Erstellt die Bibliography
	
%	\include{chapters/anhang}
	

\end{document}