\chapter*{Aufgabe 1}
\begin{center}
	\begin{tabular}	{|m{4cm}|m{6cm}|m{6cm}|}
		\hline
		Model/ Eignschaften & 1.Fehlertoleranz & 2. Parallelisierbarkeit  \\
		\hline 
		Layered & \underline{Schlecht:} Lange Startzeit der Schichten führt zu langer Ausfallzeit &  \underline{Schlecht:} Parallelisierung von der Architektur nicht direkt unterstützt. \\
		\hline 
		Model-View-Controller& \underline{Schlecht:}
		Alle drei Teile sind für den betrieb wichtig. Ohne View keine befehle an Controller und ohne Controller keine Änderungen im Modell und ohne Modell keine Daten.
		\underline{Mittel:} Bei mehreren Views kann eine Ausfallen aber die anderen die mit dem Model gekoppelt sind laufen weiter
		
				 & \underline{Mittel:} Auf ein Modell können mehrere Views zugreifen und upgedated werden. Modell kann auch wiederverwendet werden für andere Controler oder Views.\\
		\hline 
		Service Based& 
		\underline{Gut:} Bei Ausfall fällt nur ein einzelner Service aus und dieser kann schnell wieder gestartet werden & \underline{Gut:} Aufgaben sind schon in kleine Pakete Aufgeteil. Unterschiedliche Services können so leicht parallelisiert werden. \\
		\hline 
	\end{tabular}
\end{center}
