\chapter{Aufgabe}
\section{}
Die Klasse \texttt{Book} ist direkt abhängig von:
\begin{enumerate}
	\item Klasse \texttt{Client}
	\item Klasse \texttt{Date}
	\item Klasse \texttt{CSVExporter}
	\item Klasse \texttt{String}
	\item Klasse \texttt{IOException}
	\item Interface \texttt{LibraryItem}

\end{enumerate}

\section{}
Die Kopplung ist mit 6 für eine so allgemein gebräuchliche Klasse wie \texttt{Book} sehr hoch.
Änderungen in den all den anderen Klassen / Interfaces müssten in \texttt{Book} berücksichtigt werden. 
Ein Buch kann als Objekt jedoch auch abseits einer Bibliotheks-Implementation oft Anwendung finden, was eine Wiederverwendung sehr wahrscheinlich macht.
\\ 
Unter dem Aspekt des Responsibility-Driven Designs macht es zudem mehr Sinn, die nicht buch-spezifischen Funktionalitäten auszulagern.
Unserer Ansicht nach wären getrennte Klassen für das Exportieren, das Buch mit seinen String-Attributen und für die bibliotheks-spezifischen Dinge wie Verfügbarkeit etc. wünschenswerter.  