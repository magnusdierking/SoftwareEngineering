%!TEX TS-program = lualatex
\documentclass[
	ngerman,
	twoside,
	pdfa=false,
	ruledheaders=section,%Ebene bis zu der die Überschriften mit Linien abgetrennt werden, vgl. DEMO-TUDaPub
	class=report,% Basisdokumentenklasse. Wählt die Korrespondierende KOMA-Script Klasse
	thesis={type=Übung},% Dokumententyp Thesis, für Dissertationen siehe die Demo-Datei DEMO-TUDaPhd
	accentcolor=TUDa-9d,% Auswahl der Akzentfarbe
	custommargins=false,% Ränder werden mithilfe von typearea automatisch berechnet
	marginpar=false,% Kopfzeile und Fußzeile erstrecken sich nicht über die Randnotizspalte
	%BCOR=5mm,%Bindekorrektur, falls notwendig
	parskip=half-,%Absatzkennzeichnung durch Abstand vgl. KOMA-Sript
	fontsize=11pt,%Basisschriftgröße laut Corporate Design ist mit 9pt häufig zu klein
%	logofile=tuda_logo.pdf, %Falls die Logo Dateien nicht installiert sind
]{tudapub}

% Sprachanpassung
\usepackage[english, main=ngerman]{babel}
\usepackage[autostyle]{csquotes}
\usepackage{microtype}
\usepackage{listings}

% Literaturverzeichnis
\usepackage{biblatex}
\addbibresource{literature.bib}

% Pakete-Mathematik & mehr
\usepackage{mathtools}
\usepackage{amsmath}
\usepackage{amsfonts}
\usepackage{subcaption}\usepackage{xcolor}

\begin{document}
	\title{Software Engineering Übung 03}
	\subtitle{Domänenmodellierung}
	\author[J.Lippert \and M. Dierking]
	{Jonathan Lippert \and Magnus Dierking}
	%\reviewer{}
	%\department{ce}

	
	\submissiondate{20.11.2020}
	

	\maketitle
	\pagenumbering{gobble}


	% Kurzzusammenfassung
%	\include{chapters/zusammenfassung}

	% Inhaltsverzeichnis 
%	\tableofcontents
	\newpage
	\pagenumbering{arabic}
	\setcounter{page}{1}

%-------------------------------------------------------------------------------------------------
     \chapter*{Aufgabe 1}
    Die Ticketagentur verkauft \textcolor{blue}{Eintrittskarten (Tickets)} für unterschiedliche Veranstaltungen. \textcolor{blue}{Veranstaltungen} haben
    einen \textcolor{blue}{Titel}, eine \textcolor{blue}{Beschreibung} und kennen \textcolor{blue}{Datum} und \textcolor{blue}{Ort} an dem sie stattfinden. Je nach Veranstaltung ist das
    \textcolor{blue}{Ticketkontingent} unterschiedlich groß. Veranstaltungen können im System auch ein leeres Ticketkontingent haben.
    Jedes Ticket hat einen \textcolor{blue}{Preis}, eine \textcolor{blue}{Ticketnummer} und ist für genau eine Veranstaltung gültig. Neben normalen
    Tickets gibt es \textcolor{blue}{Gruppentickets} und \textcolor{blue}{personalisierte Tickets}. Gruppentickets kennen die \textcolor{blue}{Anzahl der Personen}, die mit
    dem Ticket Zutritt zur Veranstaltung bekommen. Personalisierte Tickets sind genau je einer Person zugeordnet.
    Natürlich können verschiedene personalisierte Tickets, derselben Person zugeordnet sein. Personen haben einen
    \textcolor{blue}{Namen} und eine \textcolor{blue}{Adresse}.
    \textcolor{blue}{Käufer/innen} sind Personen und müssen im System registriert sein, um Tickets kaufen zu können. Von ihnen ist
    außerdem noch die \textcolor{blue}{Kreditkartennummer} bekannt. Käufer/innen können personalisierte Tickets auch für andere
    Personen (und nicht unbedingt nur für sich selber) kaufen.
    Die Ticketagentur bietet außerdem \textcolor{blue}{Veranstaltungskataloge} für unterschiedliche \textcolor{blue}{Genre} an. Ein solcher Katalog muss
    auf jeden Fall mehr als 10 Veranstaltungen enthalten. Jede Veranstaltung wird in mindestens einem Katalog gelistet.
	
%-------------------------------------------------------------------------------------------------

	% Literaturverzeichnis
	\printbibliography % Erstellt die Bibliography
	
%	\include{chapters/anhang}
	

\end{document}