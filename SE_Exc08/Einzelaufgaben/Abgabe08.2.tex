\lstset{language=Java,
	showspaces=false,
	showtabs=false,
	breaklines=true,
	showstringspaces=false,
	breakatwhitespace=true,
}

\definecolor{dkgreen}{rgb}{0,0.6,0}
\definecolor{gray}{rgb}{0.5,0.5,0.5}
\definecolor{mauve}{rgb}{0.58,0,0.82}

\lstset{	
	frame=tb,
	language=Java,
	aboveskip=3mm,
	belowskip=3mm,
	showstringspaces=false,
	columns=flexible,
	basicstyle={\small\ttfamily},
	numbers=none,
	numberstyle=\tiny\color{gray},
	keywordstyle=\color{blue},
	commentstyle=\color{dkgreen},
	stringstyle=\color{mauve},
	breaklines=true,
	breakatwhitespace=true,
	tabsize=3,
	backgroundcolor=\color{black!5},
	numbers=left, stepnumber=1, numberstyle = \tiny
	% set backgroundcolor
}

\chapter{MCDC Testabdeckung}
\section{a)}
\begin{center}
	\begin{tabular}{ c c c c c c c }
		\hline
		Testeingabe & Erwartetes Ergebnis/Exception & c1 & c2 & c3 & c4 & Decision  \\
		\hline 
		 hexaDigitSum(?)& false & false & true & true & false & false  \\  
		 hexaDigitSum(7)& true & false & false & true & false & true  
	\end{tabular}
\end{center}

\section{b)}

\begin{lstlisting}[caption = {Conditions der aufgabe 2b}]
if(
   currNum<0 |    // Condition c1
   (currNum>=0 &   // Condition c2
   ((currNum>9 &  // Condition c3
   currNum<'A'    // Condition c4
   )|
   currNum>'F'))    // Condition c5
)
\end{lstlisting}

Angenommen wir wollen die MCDC für c = c3 anwenden. Da die Decision einmal Wahr und einmal Falsch sein soll, aber die restlichen Conditions gleich bleiben sollen,
müssen die Oder-Verknüpften Conditions Falsch und die Und-Verknüpften Wahr ergeben. Also muss C1 = Wahr und C2= Falsch gelten. Dies ist jedoch ein Widerspruch, da sie einander disjunkte Ereignisse darstellen. Mit Hilfe des Widerspruches ist also bewiesen, dass dies nicht für alle Conditions möglich ist.


