%!TEX TS-program = lualatex
\documentclass[
	ngerman,
	twoside,
	pdfa=false,
	ruledheaders=section,%Ebene bis zu der die Überschriften mit Linien abgetrennt werden, vgl. DEMO-TUDaPub
	class=report,% Basisdokumentenklasse. Wählt die Korrespondierende KOMA-Script Klasse
	thesis={type=Übung},% Dokumententyp Thesis, für Dissertationen siehe die Demo-Datei DEMO-TUDaPhd
	accentcolor=TUDa-9d,% Auswahl der Akzentfarbe
	custommargins=false,% Ränder werden mithilfe von typearea automatisch berechnet
	marginpar=false,% Kopfzeile und Fußzeile erstrecken sich nicht über die Randnotizspalte
	%BCOR=5mm,%Bindekorrektur, falls notwendig
	parskip=half-,%Absatzkennzeichnung durch Abstand vgl. KOMA-Sript
	fontsize=11pt,%Basisschriftgröße laut Corporate Design ist mit 9pt häufig zu klein
%	logofile=tuda_logo.pdf, %Falls die Logo Dateien nicht installiert sind
]{tudapub}

% Sprachanpassung
\usepackage[english, main=ngerman]{babel}
\usepackage[autostyle]{csquotes}
\usepackage{microtype}
\usepackage{listings}
\usepackage{array}

% Literaturverzeichnis
\usepackage{biblatex}
\addbibresource{literature.bib}

% Pakete-Mathematik & mehr
\usepackage{mathtools}
\usepackage{amsmath}
\usepackage{amsfonts}
\usepackage{subcaption}
\usepackage{color}

% neu cmds
\newcommand*\diff{\mathop{}\!\mathrm{d}}
\newcommand*\Diff[1]{\mathop{}\!\mathrm{d^#1}}


\begin{document}
	\title{Software Engineering Übung 08}
	\subtitle{Verifikation}
	\author[J. Lippert \and M. Dierking]
	{Jonathan Lippert \and Magnus Dierking}
	%\reviewer{}
	%\department{ce}

	
	\submissiondate{\today}
	

	\maketitle
	\pagenumbering{gobble}


	% Kurzzusammenfassung
%	\include{chapters/zusammenfassung}

	% Inhaltsverzeichnis 
%	\tableofcontents
	\newpage
	\pagenumbering{arabic}
	\setcounter{page}{1}
	

%-------------------------------------------------------------------------------------------------
     \chapter{Systematisches Testen von Methoden}
\section{Branch-Coverage}
\begin{center}
	\begin{tabular}{ c c }
		\hline
		Testeingabe & Erwartetes Ergebnis/Exception \\
		\hline 
		&    \\  
		&    \\   
		&    \\
		&    \\
		&   
	\end{tabular}
\end{center}


\section{Condition-Coverage}
\begin{center}
	\begin{tabular}{ c c }
		\hline
		Testeingabe & Erwartetes Ergebnis/Exception \\
		\hline 
		&    \\  
		&    \\   
		&    \\
		&    \\
		&   
	\end{tabular}
\end{center}
     \lstset{language=Java,
	showspaces=false,
	showtabs=false,
	breaklines=true,
	showstringspaces=false,
	breakatwhitespace=true,
}

\definecolor{dkgreen}{rgb}{0,0.6,0}
\definecolor{gray}{rgb}{0.5,0.5,0.5}
\definecolor{mauve}{rgb}{0.58,0,0.82}

\lstset{	
	frame=tb,
	language=Java,
	aboveskip=3mm,
	belowskip=3mm,
	showstringspaces=false,
	columns=flexible,
	basicstyle={\small\ttfamily},
	numbers=none,
	numberstyle=\tiny\color{gray},
	keywordstyle=\color{blue},
	commentstyle=\color{dkgreen},
	stringstyle=\color{mauve},
	breaklines=true,
	breakatwhitespace=true,
	tabsize=3,
	backgroundcolor=\color{black!5},
	numbers=left, stepnumber=1, numberstyle = \tiny
	% set backgroundcolor
}

\chapter{MCDC Testabdeckung}
\section{a)}
\begin{center}
	\begin{tabular}{ c c c c c c c }
		\hline
		Testeingabe & Erwartetes Ergebnis/Exception & c1 & c2 & c3 & c4 & Decision  \\
		\hline 
		 hexaDigitSum(?)& false & false & true & true & false & false  \\  
		 hexaDigitSum(7)& true & false & false & true & false & true  
	\end{tabular}
\end{center}

\section{b)}

\begin{lstlisting}[caption = {Conditions der aufgabe 2b}]
if(
   currNum<0 |    // Condition c1
   (currNum>=0 &   // Condition c2
   ((currNum>9 &  // Condition c3
   currNum<'A'    // Condition c4
   )|
   currNum>'F'))    // Condition c5
)
\end{lstlisting}

Angenommen wir wollen die MCDC für c = c3 anwenden. Da die Decision einmal Wahr und einmal Falsch sein soll, aber die restlichen Conditions gleich bleiben sollen,
müssen die Oder-Verknüpften Conditions Falsch und die Und-Verknüpften Wahr ergeben. Also muss C1 = Wahr und C2= Falsch gelten. Dies ist jedoch ein Widerspruch, da sie einander disjunkte Ereignisse darstellen. Mit Hilfe des Widerspruches ist also bewiesen, dass dies nicht für alle Conditions möglich ist.




	
%-------------------------------------------------------------------------------------------------

	% Literaturverzeichnis
	\printbibliography % Erstellt die Bibliography
	
%	\include{chapters/anhang}
	

\end{document}