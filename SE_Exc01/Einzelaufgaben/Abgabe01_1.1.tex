\chapter*{Aufgabe 1}
    \section*{Interactor viewpoints}
        \subsection*{Mensapersonal}
            Das Personal der Mensa will mit dem System das Guthaben jedes Studenten verwalten können. Hierzu zählt der lesende und schreibende Zugriff auf das jeweilige Guthaben. Zudem sollen Karten im Verlustfall gesperrt und gegen Vorlage eines Ausweises und einer Universitätsangehörigkeitsbescheinigung wieder freigeschaltet werden können. 
            
        \subsection*{Bibliothekspersonal}
            Das Personal der Bibliothek will mit dem System lesend und schreibend auf das Ausleihen-Konto eines Studenten zugreifen können. Zudem sollen Karten im Verlustfall gesperrt und gegen Vorlage eines Ausweises und einer Universitätsangehörigkeitsbescheinigung wieder freigeschaltet werden können.
        \subsection*{Gebäudepersonal} 
            Das Gebäudepersonal soll einzelnen Personen Schließberechtigungen für Gebäude und Räume verteilen und entziehen können. Zudem sollen sie selbst alle erteilten Berechtigungen der ihnen zugeteilten Gebäude einsehen können. Zudem sollen Karten im Verlustfall gesperrt und gegen Vorlage eines Ausweises und einer Universitätsangehörigkeitsbescheinigung wieder freigeschaltet werden können.
        \subsection*{Universitätsangehörige}
            Ein/Eine Angehöriger/Angehörige der Universität soll in der Lage sein Bücher auszuleihen, Essen in der Mensa zu bezahlen und Zutritt zu Gebäuden zu erhalten, zu dessen Zutritt er/sie berechtigt sind.
        
    \section*{Indirect viewpoints}
        \subsection*{Finanzdezernat}
            Das Finanzdezernat fordert als Rahmenbedingung, dass alle steuerrechtlich relevanten Geldtransaktionen der Karten gespeichert werden und automatisch in das bereits vorhandenen Accountingsystem der Universität übertragen werden. 
        \subsection*{Gesetzgeber}  
            Der Gesetzgeber fordert als Rahmenbedingung die Einhaltung der geltenden Datenschutzgesetze. Hierzu soll es keinem der Interactors möglich sein Zugriff auf einen ihm nicht zugeteilten Bereich zu erhalten. Zudem müssen die vorhandenen Schließlogs nach 3 Wochen gelöscht werden. 
            %%%%%%%%%%%%%%%%%%%%%%%%%%