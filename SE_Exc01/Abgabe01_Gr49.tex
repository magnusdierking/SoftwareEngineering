%!TEX TS-program = lualatex
\documentclass[
	ngerman,
	twoside,
	pdfa=false,
	ruledheaders=section,%Ebene bis zu der die Überschriften mit Linien abgetrennt werden, vgl. DEMO-TUDaPub
	class=report,% Basisdokumentenklasse. Wählt die Korrespondierende KOMA-Script Klasse
	thesis={type=Übung},% Dokumententyp Thesis, für Dissertationen siehe die Demo-Datei DEMO-TUDaPhd
	accentcolor=TUDa-9d,% Auswahl der Akzentfarbe
	custommargins=false,% Ränder werden mithilfe von typearea automatisch berechnet
	marginpar=false,% Kopfzeile und Fußzeile erstrecken sich nicht über die Randnotizspalte
	%BCOR=5mm,%Bindekorrektur, falls notwendig
	parskip=half-,%Absatzkennzeichnung durch Abstand vgl. KOMA-Sript
	fontsize=11pt,%Basisschriftgröße laut Corporate Design ist mit 9pt häufig zu klein
%	logofile=tuda_logo.pdf, %Falls die Logo Dateien nicht installiert sind
]{tudapub}

% Sprachanpassung
\usepackage[english, main=ngerman]{babel}
\usepackage[autostyle]{csquotes}
\usepackage{microtype}
\usepackage{listings}

% Literaturverzeichnis
\usepackage{biblatex}
\addbibresource{literature.bib}

% Pakete-Mathematik & mehr
\usepackage{mathtools}
\usepackage{amsmath}
\usepackage{amsfonts}
\usepackage{subcaption}


% neu cmds
\newcommand*\diff{\mathop{}\!\mathrm{d}}
\newcommand*\Diff[1]{\mathop{}\!\mathrm{d^#1}}


\begin{document}
	\title{Software Engineering Übung 01}
	\subtitle{Anforderungsanalyse}
	\author[J.Lippert \and M. Dierking]
	{Jonathan Lippert \and Magnus Dierking}
	%\reviewer{}
	%\department{ce}

	
	\submissiondate{20.11.2020}
	

	\maketitle
	\pagenumbering{gobble}


	% Kurzzusammenfassung
%	\include{chapters/zusammenfassung}

	% Inhaltsverzeichnis 
%	\tableofcontents
	\newpage
	\pagenumbering{arabic}
	\setcounter{page}{1}

%-------------------------------------------------------------------------------------------------
     \chapter*{Aufgabe 1}
    \section*{Interactor viewpoints}
        \subsection*{Mensapersonal}
            Das Personal der Mensa will mit dem System das Guthaben jedes Studenten verwalten können. Hierzu zählt der lesende und schreibende Zugriff auf das jeweilige Guthaben. Zudem sollen Karten im Verlustfall gesperrt und gegen Vorlage eines Ausweises und einer Universitätsangehörigkeitsbescheinigung wieder freigeschaltet werden können. 
            
        \subsection*{Bibliothekspersonal}
            Das Personal der Bibliothek will mit dem System lesend und schreibend auf das Ausleihen-Konto eines Studenten zugreifen können. Zudem sollen Karten im Verlustfall gesperrt und gegen Vorlage eines Ausweises und einer Universitätsangehörigkeitsbescheinigung wieder freigeschaltet werden können.
        \subsection*{Gebäudepersonal} 
            Das Gebäudepersonal soll einzelnen Personen Schließberechtigungen für Gebäude und Räume verteilen und entziehen können. Zudem sollen sie selbst alle erteilten Berechtigungen der ihnen zugeteilten Gebäude einsehen können. Zudem sollen Karten im Verlustfall gesperrt und gegen Vorlage eines Ausweises und einer Universitätsangehörigkeitsbescheinigung wieder freigeschaltet werden können.
        \subsection*{Universitätsangehörige}
            Ein/Eine Angehöriger/Angehörige der Universität soll in der Lage sein Bücher auszuleihen, Essen in der Mensa zu bezahlen und Zutritt zu Gebäuden zu erhalten, zu dessen Zutritt er/sie berechtigt sind.
        
    \section*{Indirect viewpoints}
        \subsection*{Finanzdezernat}
            Das Finanzdezernat fordert als Rahmenbedingung, dass alle steuerrechtlich relevanten Geldtransaktionen der Karten gespeichert werden und automatisch in das bereits vorhandenen Accountingsystem der Universität übertragen werden. 
        \subsection*{Gesetzgeber}  
            Der Gesetzgeber fordert als Rahmenbedingung die Einhaltung der geltenden Datenschutzgesetze. Hierzu soll es keinem der Interactors möglich sein Zugriff auf einen ihm nicht zugeteilten Bereich zu erhalten. Zudem müssen die vorhandenen Schließlogs nach 3 Wochen gelöscht werden. 
            %%%%%%%%%%%%%%%%%%%%%%%%%%
     \chapter*{Aufgabe 2}
\section*{1.}
    Die gegebene Anforderung ist ungeeignet. Eine detaillierte Klarstellung, welche Informationen als relevant gelten muss der Anforderung hinzugefügt werden.
    "Die Personaltrainer-innen des Fitnessstudios dürfen nur Informationen über Vertragslaufzeit, Trainingswerte und gebuchte Kurse der Trainierenden
    einsehen."
\section*{2.}
    Die gegebene Anforderung ist geeignet.
\section*{3.}
   Die gegebene Anforderung ist ungeeignet. Die Regelmäßigkeit des Abgleiches muss weiter spezifiziert werden.
   "Das Kassensystem (z.B. eines Supermarktes) muss sich alle 30 Minuten mit dem Inventarsystem abgleichen."
	
%-------------------------------------------------------------------------------------------------

	% Literaturverzeichnis
	%\printbibliography % Erstellt die Bibliography
	
%	\include{chapters/anhang}
	

\end{document}